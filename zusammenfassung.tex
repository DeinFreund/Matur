\documentclass[8pt,a4paper]{article}
\usepackage[a4paper,vmargin={30mm,20mm}]{geometry}
\usepackage [ngerman,english] {babel}
\usepackage [utf8] {inputenc}
\usepackage{amsmath,amsfonts,amssymb}
\usepackage [pdftex] {graphicx}
\title {Application to the MIT / RSI 2014}
\author {Fabian Lyck}
\date {\today}
\begin{document}

\section{"Zusammenfassung"}
Es geht darum, ein Multiplayer-Spiel zu programmieren, das auf dem Server läuft und somit auch unabhängig davon ist, ob ein Client überhaupt eingeloggt ist. Es spielt im Weltraum und beinhaltet somit Gravitation und Himmelsobjekte. Man selber steuert ein Raumschiff, welches verschiedene Slots besitzt, an denen man Module befestigen kann(zB kann man Antriebe befestigen um das Schiff in verschiedene Richtungen zu beschleunigen, dies ist bereits implementiert, wird jedoch für die Arbeit revidiert). Ein Zusatz wäre eine Scriptsprache, die automatisierte Schiffe ermöglicht. Das Spiel läuft auf der Unity Engine und ist somit 3dimensional.
\section{"Aufgaben"}
\subsection{"Klassendesign"}
\subsection{"Implementierung"}
\subsection{"Applikationsprotokoll"}
\subsection{"Dokumentation"}
\subsection{"Erarbeitung von Theorien"}
\subsection{"Design GUI"}
\subsection{"Design 3D"}
\subsection{"Terminplan"}
\subsection{"Themen des Maturberichts"}

\end{document}
